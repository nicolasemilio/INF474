\documentclass[12pt,spanish,letter]{article}

\usepackage[latin1]{inputenc}% Tildes
\usepackage[spanish]{babel}
\usepackage{graphicx}
\usepackage{amsmath, amsthm, amsfonts}
\usepackage[small,bf]{caption}
\setlength{\captionmargin}{20pt}
\usepackage{indentfirst}
\usepackage{multicol}
\usepackage{enumerate}
\usepackage{booktabs}
\usepackage{array}
\newcolumntype{?}{!{\vrule width 2pt}}


\usepackage{authblk}
\renewcommand\Authand{  }
\renewcommand\Authands{,  }

\usepackage{hyperref}
\usepackage[none]{hyphenat}
\tolerance=6000

\usepackage{geometry}
\geometry{right=1in, left=1in}
\geometry{top=1in, bottom=1in}
\geometry{includefoot}
\textheight = 8.5in

\usepackage{tikz}
\usetikzlibrary{shapes,arrows}
\tikzstyle{line} = [draw, ->, line width=1pt]
\tikzstyle{arrow} = [draw, ->,  line width=1pt, thick, dash dot]

\usepackage{verbatim}
\usepackage{enumitem}


\usepackage{listings}
\lstdefinestyle{customC}{
  %language=C,
  showstringspaces=false,
  basicstyle=\small\ttfamily,
  keywordstyle=\bfseries\color{blue},
  commentstyle=\itshape\color{white!40!black},
  tabsize = 4,
  stringstyle=\color{red},
}


\usepackage{fancyhdr}
\pagestyle{fancy}
\usepackage{lastpage}

\fancypagestyle{plain}
{
  \setlength{\topmargin}{-0.5in}
  \setlength{\headheight}{1.05in}
  \setlength{\headsep}{-0.0in}
  \setlength{\headwidth}{\textwidth}
  \setlength{\footskip}{2.0in}
  \fancyhead[l]{
    {\bf T\'opicos Avanzados en Inteligencia Artificial}\\
    {\bf Tarea Pr\'actica 1}
  }
  \fancyhead[r]{
    \includegraphics[scale=0.15]{src/usm_logo}
  }
}
%====================================

\pagestyle{myheadings}
\markright{\bf{T\'opicos Avanzados en Inteligencia Artificial}}


\title{\Large \bf
Inteligencia Artificial\\ Tarea Pr\'actica 1}
\author{Profesor: Nicol\'as Rojas Morales\\}
\affil{\textit{Universidad T\'ecnica Federico Santa Mar\'ia, \\ Departamento de Inform\'atica. }}
\date{\normalsize }


\begin{document}

\thispagestyle{fancy}
\maketitle
\thispagestyle{plain}
\let\oldthefootnote\thefootnote
\renewcommand{\thefootnote}{\fnsymbol{footnote}}
\let\thefootnote\oldthefootnote
\setlength{\headheight}{0.65in}
\setlength{\textheight}{8.60in}
\pagestyle{myheadings}

\renewcommand{\tablename}{Tabla} 
\renewcommand{\figurename}{{\bf \emph{Figura}}} 
\maketitle

\section{Ejecutando un Algoritmo}

La primera tarea busca ejercitar la creaci\'on, dise\~no y ejecuci\'on de un escenario experimental en su m\'aquina.
El objetivo de este escenario
sera visualizar el desempe\~no de un algoritmo (Ant Knapsack)
en un set de instancias, tanto en la calidad de las soluciones que puede obtener como en su tiempo de ejecuci\'on. \\

Ant Knapsack (AK) es un algoritmo de hormigas dise\~nado para la resoluci\'on del Multidimensional Knapsack Problem.
AK requiere los siguientes argumentos para su ejecuci\'on (puede observarlos en Initialize.cpp):
\begin{verbatim}
InstanciaFile = argv[1]; \\ Path a la instancia
Seed = atoi(argv[2]); \\ Semilla 
TotalAnts = atoi(argv[3]); \\ Cantidad de hormigas
TotalEvaluations = atoi(argv[4]); \\ Cantidad de Evaluaciones (Recursos)
alpha = atof(argv[5]); \\ Importancia de la Feromona
beta = atof(argv[6]); \\ Importancia del Conocimiento Heuristico
ph_max = atof(argv[7]); \\ Feromona maxima en la matriz
ph_min = atof(argv[8]); \\ Feromona minima en la matriz
rho = atof(argv[9]); \\ Tasa de Evaporacion de la Feromona
\end{verbatim}

Un ejemplo de ejecuci\'on de AK:

\begin{verbatim}
./AK ../instances/mknap1_7.txt 1 20 10000 9 1 50 0.01 0.99
\end{verbatim}

considerando el orden de los argumentos y par\'ametros se\~nalado, AK obtiene como resultado:

\begin{verbatim}
0.16327
\end{verbatim}

Esto significa que la calidad de la mejor soluci\'on encontrada por AK se encuentra a un 0.16\% de distancia al \'optimo global de la instancia MKNAP1\_7. El gap porcentual se calcula (observar en Main.cpp):

\begin{verbatim}
cout << 100*(Opt - BestFitnessFound)/Opt << endl;
\end{verbatim}



\subsection{Instancias}
Para esta tarea, utilizaremos las siguientes instancias benchmark:

\begin{verbatim}
OR10x500-0.25_8.dat
OR5x500-0.25_7.dat
mknap1_5.txt
mknap1_7.txt
OR30x500-0.25_3.dat
gk01.dat
mknap1_6.txt	
sento1.txt
\end{verbatim}

Mas detalles pueden ser obtenidos en \url{https://www.researchgate.net/publication/271198281_Benchmark_instances_for_the_Multidimensional_Knapsack_Problem}

\section{Detalles Entrega}

Se requiere crear un script que realice ejecuciones independientes de AK en todas las instancias. Considere:

\begin{itemize}
\item Se deben realizar 20 ejecuciones independientes (semillas distintas) por instancia.
\item Los valores a los par\'ametros de AK ser\'an le\'idos de un archivo de texto llamado params.txt:
\begin{verbatim}
20 10000 9 1 50 0.01 0.99
\end{verbatim}
\item Mientras se ejecuta AK, el script debe crear un archivo de salida por instancia, donde se almacene la calidad de la mejor soluci\'on obtenida por AK (por semilla).
\item Adem\'as, el script debe almacenar el tiempo de ejecuci\'on de AK (en segundos) en otro archivo de texto.
\item Para agilizar la ejecuci\'on de los experimentos, el script puede controlar la cantidad de ejecuciones simult\'aneas de AK con un par\'ametro (cantidad de hebras disponibles para ejecutar).
\item Una vez terminada la ejecuci\'on de AK, el script debe obtener m\'etricas de desempe\~no (mejor obtenido, $\mu$, $\sigma$, tiempo promedio de ejecuci\'on, minimo y m\'aximo tiempo de ejecuci\'on).
\item El script debe generar un archivo que contenga una tabla con las m\'etricas de desempe\~no (por instancia) en formato \LaTeX.
\end{itemize}

Adem\'as, es requisito entregar un escrito en ingl\'es que contenga:
\begin{itemize}
\item Una secci\'on que presente caracter\'isticas sobre las instancias utilizadas.
\item Explicaci\'on del escenario experimental y hardware utilizado. Detallar valores de par\'ametros utilizados, cantidad de recursos (n\'umero de evaluaciones).
\item Resultados obtenidos por AK en cada instancia y an\'alisis respectivo.
\item El escrito debe ser realizado en \LaTeX.
\end{itemize}

\textbf{Evaluaci\'on de su script}:
\begin{itemize}
\item Su script ser\'a revisado en una m\'aquina Linux/Unix.
\item Considere que su script se ejecutar\'a en un directorio que adem\'as contendr\'a el archivo de par\'ametros, el directorio AntKnapsack y el directorio de instancias (disponibles en Aula).
\item Se sugiere realizar make clean y make antes de ejecutar sus experimentos.
\end{itemize}

\textbf{Fecha de Entrega: 31 de Agosto - Informe en Clases - Script en Aula (Nombre\_Apellido.zip).}
\end{document}          
